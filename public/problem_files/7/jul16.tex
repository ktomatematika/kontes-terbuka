\documentclass[a4paper,12pt]{article}
\usepackage{graphicx, hyperref, amsmath, amssymb, amsthm, amsfonts, gensymb, pgf, tikz}
\usepackage[a4paper, margin=1in]{geometry}
\usepackage[bahasa]{babel}

\usepackage{fancyhdr}
\rfoot{Halaman \thepage}
\cfoot{}
\lfoot{Kontes Terbuka Olimpiade Matematika}
\renewcommand{\headrulewidth}{0pt} % no line in header area
\renewcommand{\footrulewidth}{1pt} % draw line in footer

\usepackage{tkz-euclide}
\usetkzobj{all}

\hyphenation{bu-kan bu-kan-lah Cheryl de-ngan di-go-long-kan di-gu-na-kan di-ka-te-go-ri-kan di-la-ku-kan di-mi-sal-kan di-se-rah-kan di-sim-pul-kan di-tun-juk-kan eks-klu-si eks-pre-si fak-tor-kan ke-se-ba-ngun-an kon-tes-tan ling-ka-ran me-li-bat-kan mem-fak-tor-kan me-mi-ni-mum-kan mem-per-da-lam me-nan-da-kan meng-a-ki-bat-kan me-nge-nai me-nger-ja-kan me-nger-ja-kan-nya me-nger-ti meng-gu-na-kan meng-in-ter-pre-ta-si me-ngu-rang-kan me-nya-ta-kan me-ru-pa-kan ob-ser-va-si pa-pan pem-fak-tor-an pe-nger-ja-an pe-rem-pu-an per-ha-ti-kan sub-sti-tu-si ten-tu-kan ter-ak-hir ter-ka-lah-kan vo-lume}

\newcommand{\lp}{\left(}
\newcommand{\rp}{\right)}
\newcommand{\lb}{\left\{}
\newcommand{\rb}{\right\}}
\newcommand{\lf}{\left\lfloor}
\newcommand{\rf}{\right\rfloor}
\newcommand{\lc}{\left\lceil}
\newcommand{\rc}{\right\rceil}
\newcommand{\ls}{\left[}
\newcommand{\rs}{\right]}

\begin{document}

\pagestyle{empty}

\begin{center}
%\includegraphics[height=100pt]{logo.png}

\vspace{10em}

{\LARGE Kontes Terbuka Olimpiade Matematika}

\vspace{1em}

{\Large Kontes Bulanan Juli 2016}

\vspace{3em}

{\large 22--25 Juli 2016}

\vspace{20em}

{\Large Berkas Soal}
\end{center}
\clearpage
\setcounter{page}{1}

\newpage

\pagestyle{fancy}

\begin{center}
\textbf{Bagian A}
\end{center}
\noindent {\it Tuliskan jawaban akhir setiap soal dengan mengisi formulir di \href{http://www.bit.ly/ktom-A-jun-16}{\sffamily bit.ly/ktom-A-jul-16}. Setiap soal bernilai 1 angka. Tidak ada pengurangan nilai atas jawaban yang salah. Jawaban soal-soal bagian A dipastikan merupakan bilangan bulat.}
\begin{enumerate}
%%% START Bagian A
\item Henry, Ilhan, Johan, dan empat orang lainnya mengikuti suatu perlombaan. Pada akhir perlombaan, masing-masing dari ketujuh orang tersebut diberi peringkat dari 1, 2, $\ldots$, sampai 7. Jika diketahui bahwa peringkat Johan lebih tinggi daripada peringkat Ilhan, dan peringkat Ilhan lebih tinggi daripada peringkat Henry, tentukan banyaknya susunan peringkat yang mungkin.

\item Diketahui bahwa jumlah dari 10 buah bilangan prima berurutan adalah $x$, yang merupakan bilangan ganjil. Tentukan nilai terbesar yang mungkin bagi $x$.

\item Diberikan sebuah segitiga $ABC$ dengan $D$ adalah titik tengah $BC$. Diketahui bahwa $\angle ADC=60^\circ$, $AB=10$, dan $AC=8$. Jika luas segitiga $ABC$ adalah $x\sqrt{y}$, dengan $x$ dan $y$ adalah bilangan asli dan $y$ tidak habis dibagi oleh kuadrat dari bilangan prima apa pun, tentukan nilai dari $x+y$.

\item Tentukan jumlah semua bilangan dua angka yang memenuhi selisih kuadrat bilangan tersebut dengan kuadrat bilangan yang diperoleh dengan membalikkan kedua angka dari bilangan tersebut adalah $1584$. (Catatan: $\overline{0a}$ sama dengan bilangan satu angka $\overline{a}$).

\item Diberikan sebuah segitiga $ABC$ dengan $BE$ dan $CF$ adalah dua garis tingginya. Apabila $\angle BAC = 60^{\circ}$ dan $p(XYZ)$ menyetakan keliling segitiga $XYZ$, tentukan nilai dari $210 \times \frac{p(AEF)}{p(ABC)}$.

\item Untuk setiap bilangan bulat positif $k$, misalkan $\alpha_k$ adalah bilangan real positif yang memenuhi persamaan $x^2-kx-1=0$. Tentukan bilangan bulat positif $n$ terkecil sedemikian sehingga $\alpha_1+\alpha_2+\cdots+\alpha_n\geq 2016$.

\item Misalkan $P$ adalah sebuah segi-banyak beraturan dengan $n \ge 4$ sisi. Empat titik sudut berbeda $A$, $B$, $C$, dan $D$ dipilih secara acak dari segi-banyak tersebut (permutasi dihitung berbeda). Misalkan peluang bahwa garis $AB$ dan $CD$ berpotongan di dalam segi-banyak adalah $\frac{a}{b}$, dengan $a$ dan $b$ adalah bilangan asli yang memenuhi $FPB(a,b) = 1$. Tentukan nilai dari $a \times b$.

\item Misalkan $ABC$ adalah sebuah segitiga sama kaki dengan $AB=AC$ dan $\angle A=100^\circ$. Titik $D$ terletak pada sinar $AC$ sedemikian sehingga $AD=BC$. Tentukan besar $\angle ABD$ (dalam derajat).

\item Misalkan $$S = 2 \cdot 2^2 + 3 \cdot 2^3 + \cdots + 2016 \cdot 2^{2016}.$$ Jika $x$ adalah bilangan ganjil terbesar yang habis membagi $S$, dan $y$ adalah bilangan bulat terbesar sedemikian sehingga $2^y$ habis membagi $S$, tentukan nilai dari $x + y$.

\item Tentukan banyaknya pasangan bilangan bulat $(x,y)$ dengan $|x|\leq 100$ dan $|y|\leq 100$ yang memenuhi persamaan $x^2+4y=4xy+1$.

\item Diberikan segitiga lancip $ABC$ dengan panjang diameter lingkaran luar 25. $AD$, $BE$, dan $CF$ adalah garis tinggi dari segitiga tersebut. Jika keliling dari segitiga $DEF$ adalah 32, tentukan luas dari segitiga $ABC$.

\item Diberikan sebuah bilangan bulat $n$ dengan $3 \le n \le 2016$. Sebanyak $n$ bilangan bulat disusun melingkar sedemikian sehingga setiap bilangan lebih besar dari jumlah bilangan yang berada pada urutan pertama dan kedua dari sebelah kanan bilangan tersebut. Misalkan $A(n)$ menyatakan nilai terbesar dari banyaknya bilangan positif di antara $n$ bilangan tersebut. Tentukan banyaknya nilai berbeda untuk $A(n)$, untuk setiap $n$ yang mungkin.

\item Misal $\{a_n\}_{n=0}^{\infty}$ adalah barisan yang memenuhi $a_0 = 0, a_1 = 1$, dan $a_{n+2} = a_{n+1}+2a_n$ untuk semua bilangan bulat $n \geq 0$. Tentukan bilangan asli terkecil $k$ sedemikian sehingga $61 | a_k$.

\item Misalkan $x,y,z$ adalah bilangan real yang memenuhi persamaan \[x^2+2y^2+2z^2-2xy-2xz=2\sqrt{x+y-z}-x-2y+2z-\frac{5}{4}.\] Tentukan nilai dari $100(x+y+z)$.
%%% END Bagian A
\end{enumerate}

\newpage

\begin{center}
\textbf{Bagian B}
\end{center}
{\it Tuliskan jawaban beserta langkah pekerjaan Anda pada lembar jawaban yang sudah disediakan. Setiap soal bernilai 7 angka. Tidak ada pengurangan nilai atas jawaban yang salah.}
\begin{enumerate}
%%% START Bagian B
\item Diberikan sebuah segitiga lancip $ABC$. Buat titik $D$ pada $BC$ sedemikian sehingga $AD$ tegak lurus dengan $BC$ dan titik $E$ pada $CA$ sedemikian sehingga $BE$ tegak lurus dengan $CA$. Misalkan $H$ adalah titik potong $AD$ dan $BE$.
\begin{enumerate}
	\item Tunjukkan bahwa $ABDE$ merupakan segiempat tali busur (segiempat di mana keempat titik sudutnya terletak pada sebuah lingkaran).

	Ingat, Anda tidak boleh membuat asumsi tambahan, seperti $ABC$ adalah segitiga sama sisi. Jawaban yang memuat hanya gambar (meskipun akurat dan berhasil `memperlihatkan' apa yang ingin dibuktikan) bukanlah bukti yang valid; Anda harus membuktikan bahwa pernyataan tersebut benar untuk sebarang segitiga yang memenuhi kriteria pada soal.

	\item Tunjukkan bahwa $EHDC$ juga merupakan segiempat tali busur.
	\item Manfaatkan bagian (a) dan (b) untuk menunjukkan bahwa $\angle HCD=\angle HAB$.
	\item Misalkan perpanjangan $CH$ memotong $AB$ di titik $F$. Tunjukkan bahwa $CF$ tegak lurus dengan $AB$.

	Perhatikan bahwa ketiga ruas garis $AD$, $BE$, dan $CF$ selalu berpotongan di satu titik (dalam hal ini $H$). Selanjutnya, garis yang melalui $AD$ atau $BE$ atau $CF$ dinamakan garis tinggi segitiga $ABC$. Titik $H$ disebut titik tinggi segitiga $ABC$, yakni perpotongan ketiga garis tinggi segitiga $ABC$.

	\item Tentukan titik tinggi dari segitiga $ABH$. Jelaskan jawaban Anda.
	\item Tunjukkan bahwa $\angle AHB+\angle ACB=180\degree$. Apakah $AHBC$ merupakan segiempat tali busur? Jelaskan jawaban Anda.

\end{enumerate}

\item Bilangan palindrom dengan lima angka didefinisikan sebagai bilangan bulat dengan angka-angka penyusun $\overline{abcba}$, di mana $a \neq 0$. Misalkan $S$ adalah hasil jumlah semua bilangan palindrom dengan lima angka. Tentukan jumlah angka-angka penyusun dari $S$.

\item Untuk setiap bilangan asli $n$ dan bilangan real positif $x$, tunjukkan bahwa $$x^n+\dfrac{1}{x^n}-2 \ge n^2 \left(x+\frac{1}{x}-2\right).$$

\item Tentukan semua tripel bilangan prima $(p,q,r)$ sedemikian sehingga $4p+4q,4q+4r,4r+4p$ ketiganya merupakan bilangan berpangkat empat (dengan kata lain, berbentuk $x^4$ untuk sebuah bilangan asli $x$).
%%% END Bagian B
\end{enumerate}
\end{document}
