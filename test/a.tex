\documentclass[bahasa, a4paper,12pt]{article}
\usepackage{graphicx, amsmath, amssymb, amsthm, amsfonts}
\usepackage[a4paper, margin=1in]{geometry}
\usepackage{babel}

\usepackage{fancyhdr}
\rfoot{Halaman \thepage}
\cfoot{}
\lfoot{Kontes Terbuka Olimpiade Matematika}
\renewcommand{\headrulewidth}{0pt} % no line in header area
\renewcommand{\footrulewidth}{1pt} % draw line in footer

\hyphenation{bu-kan bu-kan-lah Cheryl de-ngan di-go-long-kan di-gu-na-kan di-ka-te-go-ri-kan di-la-ku-kan di-mi-sal-kan di-se-rah-kan di-sim-pul-kan di-tun-juk-kan eks-klu-si eks-pre-si fak-tor-kan ke-se-ba-ngun-an kon-tes-tan ling-ka-ran me-li-bat-kan mem-fak-tor-kan me-mi-ni-mum-kan mem-per-da-lam me-nan-da-kan meng-a-ki-bat-kan me-nge-nai me-nger-ja-kan me-nger-ja-kan-nya me-nger-ti meng-gu-na-kan meng-in-ter-pre-ta-si me-ngu-rang-kan me-nya-ta-kan me-ru-pa-kan ob-ser-va-si pa-pan pem-fak-tor-an pe-nger-ja-an pe-rem-pu-an per-ha-ti-kan sub-sti-tu-si ten-tu-kan ter-ak-hir ter-ka-lah-kan vo-lume}

\newcommand{\lp}{\left(}
\newcommand{\rp}{\right)}
\newcommand{\lb}{\left\{}
\newcommand{\rb}{\right\}}
\newcommand{\lf}{\left\lfloor}
\newcommand{\rf}{\right\rfloor}
\newcommand{\lc}{\left\lceil}
\newcommand{\rc}{\right\rceil}
\newcommand{\ls}{\left[}
\newcommand{\rs}{\right]}

\begin{document}

\pagestyle{empty}

\begin{center}
\includegraphics[height=100pt]{logo.png}

\vspace{10em}

{\LARGE Kontes Terbuka Olimpiade Matematika}

\vspace{1em}

{\Large Kontes Bulanan September 2016}

\vspace{3em}

{\large 23 -- 26 September 2016}

\vspace{20em}

{\Large Berkas Soal}
\end{center}
\clearpage
\setcounter{page}{1}

\newpage

\pagestyle{fancy}

\begin{center}
\textbf{Bagian A}
\end{center}
\noindent {\it Untuk setiap soal, tuliskan saja jawaban akhirnya. Setiap soal bernilai 1 angka. Tidak ada pengurangan nilai untuk jawaban yang salah atau dikosongkan. Jawaban soal-soal bagian A dipastikan merupakan bilangan bulat.}

\begin{enumerate}
%%% START Bagian A

\item Misalkan $x$ dan $y$ adalah dua bilangan real berbeda yang memenuhi persamaan $y+6=(x-6)^2$ dan $x+6=(y-6)^2$ secara bersamaan. Tentukan nilai dari $x^3+y^3$.

\item Terdapat dua koin identik yang lebih cenderung memunculkan angka daripada gambar. Kedua koin tersebut dilemparkan bersama-sama. Diketahui bahwa peluang munculnya tepat sebuah gambar dan sebuah angka adalah 0,48. Jika $p$ adalah peluang munculnya tepat dua buah angka, tentukan nilai $100p$.

\item Tentukan hasil penjumlahan semua bilangan asli $n$ yang kurang dari 100 dan memenuhi $100\mid n^3+n^2+n+1$.

\item Diberikan segitiga $ABC$. Garis bagi sudut $A$ memotong $BC$ di titik $D$. Garis bagi dalam $\angle ADB$ memotong $AB$ di titik $E$. Jika $BE=7$, $AE=14$, dan $DE\parallel AC$, tentukan kuadrat dari panjang $AD$.

\item  Tentukan banyaknya bilangan real positif $x$ yang memenuhi persamaan \[\lfloor{x}\rfloor=\{x\}^2+2016\{x\}+2016,\] di mana $\lfloor{x}\rfloor$ menyatakan bilangan bulat terbesar yang lebih kecil dari atau sama dengan $x$, dan $\{x\}=x-\lfloor{x}\rfloor$.

\item Tentukan banyaknya bilangan tiga-angka $\overline{abc}$ sedemikian sehingga $a$ dan $c$ keduanya bukan nol dan bilangan tiga-angka $\overline{abc}$ dan $\overline{cba}$ keduanya habis dibagi 4.

\item Terdapat dua lingkaran $\Gamma_{1}$ dan $\Gamma_{2}$ yang bersinggungan dalam di titik $A$. Lingkaran $\Gamma_{2}$ terletak di dalam lingkaran $\Gamma_{1}$. Misalkan $P$ adalah sebuah titik pada lingkaran $\Gamma_{2}$ ($P\neq A$). Diketahui bahwa garis singgung lingkaran $\Gamma_{2}$ di titik $P$ memotong lingkaran $\Gamma_{1}$ di titik $B$ dan $C$. Jika panjang $AB=81$, $BP=27$ dan $AC=54$, tentukan panjang $CP$.

\item Diketahui bahwa suatu fungsi $f: \mathbb{R} \rightarrow \mathbb{R}$ memenuhi \[ f(x^2+x+3)+2f(x^2-3x+5) = 6x^2-10x+17. \] Tentukan nilai dari $f(2016)$.

\item Misalkan $N$ menyatakan banyaknya barisan berhingga $a_1,a_2,\dotsc,a_{2016}$ dengan $a_i\in \{1,2,3\}$ untuk $i=1,2,\dotsc,2016$ dengan syarat bahwa angka 2 muncul sebanyak ganjil kali pada barisan tersebut. Jika $2N+1$ dapat dinyatakan dalam bentuk $m^n$, dengan $m$ suatu bilangan prima dan $n$ suatu bilangan bulat positif, tentukan nilai dari $m+n$.

\item Tentukan banyaknya tripel bilangan asli $(a,b,c)$ yang memenuhi persamaan \[abc=3a+3b+3c+24.\]

\item Misalkan $\triangle ABC$ memenuhi $\angle A=45^\circ$, $\angle B=60^\circ$, $\angle C=75^\circ$, dan $AC=40$. Misalkan pula $D$ dan $E$ terletak pada $BC$ dan $AC$, berturut-turut, sehingga $AD$ dan $BE$ adalah garis tinggi $\triangle ABC$. Selanjutnya, misalkan $DE$ memotong garis yang melewati $C$ dan sejajar $AB$ di titik $F$. Notasikan $M$ sebagai titik tengah sisi $AB$. Jika $CM$ memotong lingkaran luar $\triangle CDE$ di titik $G$, hitunglah kuadrat dari panjang $FG$.

\item Tentukan nilai terkecil dari fungsi \[f(a,b,c,d)=\left\lfloor\frac{a+b+c+d}{a}\right\rfloor + \left\lfloor\frac{a+b+c+d}{b}\right\rfloor + \left\lfloor\frac{a+b+c+d}{c}\right\rfloor + \left\lfloor\frac{a+b+c+d}{d}\right\rfloor,\] di mana $a$, $b$, $c$, dan $d$ adalah bilangan real positif dan $\lfloor{x}\rfloor$ menyatakan bilangan bulat terbesar yang lebih kecil dari atau sama dengan $x$.

\item Misalkan $S$ adalah himpunan semua bilangan bulat yang berbentuk $2^a + 2^b$, di mana $0 \leq a < b < 40$. Jika peluang suatu bilangan bulat dari $S$ yang dipilih secara acak habis dibagi 9 adalah $\frac{p}{q}$, di mana $p$ dan $q$ adalah bilangan bulat positif yang relatif prima, tentukan nilai dari $p + q$.

\item Sebutlah sebuah bilangan enam-angka $\overline{abcdef}$ \textit{tubis} jika bilangan tersebut habis dibagi 3, bilangan lima-angka $\overline{bcdef}$, bilangan empat-angka $\overline{cdef}$, bilangan tiga-angka $\overline{def}$, bilangan dua-angka $\overline{ef}$, dan bilangan satu-angka $\overline{f}$ semuanya tidak habis dibagi 3, dan semua angka penyusun dari bilangan $\overline{abcdef}$ tidak bernilai nol. Tentukan banyaknya bilangan enam-angka tubis.

%%% END Bagian A
\end{enumerate}

\newpage

\begin{center}
\textbf{Bagian B}
\end{center}
{\it Tuliskan jawaban beserta langkah pekerjaan Anda secara lengkap. Jawaban boleh diketik, difoto, ataupun di-scan. Setiap soal bernilai 7 angka. Tidak ada pengurangan nilai untuk jawaban yang salah.}
\begin{enumerate}
%%% START Bagian B
\item Diberikan bilangan-bilangan real positif $a$, $b$, dan $c$.
\begin{enumerate}
	\item[(i)] Tunjukkan bahwa $\dfrac{a}{b}+\dfrac{b}{c}+\dfrac{c}{a}\geq 3$. \\
	(\textit{Petunjuk: manfaatkan ketaksamaan AM-GM})
	\item[(ii)] Tunjukkan bahwa
	\begin{equation}\label{al0}
	\frac{a}{b}\geq \frac{2a}{1+b^2}\geq 2a-ab.\tag{*}
	\end{equation}
\end{enumerate}
	Perhatikan bahwa dengan menyubstitusi $(a,b)$ dengan $(b,c)$ dan $(c,a)$ ke  ketaksamaan (\ref{al0}), dan dilanjutkan dengan menjumlahkan kedua ketaksamaan yang didapat dengan ketaksamaan (\ref{al0}), kita peroleh
	\begin{equation*}\label{al1}
	\frac{a}{b}+\frac{b}{c}+\frac{c}{a} \geq \frac{2a}{1+b^2}+\frac{2b}{1+c^2}+\frac{2c}{1+a^2}\geq 2(a+b+c)-(ab+bc+ca).\tag{**}
	\end{equation*}
	\begin{enumerate}
	\item[(iii)] Perhatikan ketaksamaan (i) dan (\ref{al1}). Apakah dapat diambil kesimpulan bahwa $\dfrac{2a}{1+b^2}+\dfrac{2b}{1+c^2}+\dfrac{2c}{1+a^2}\geq 3$ untuk setiap bilangan real positif $a$, $b$, dan $c$? Jika dapat, buktikan. Jika tidak, berikan contoh $a$, $b$, $c$ yang menyangkal pernyataan tersebut.
	\item[(iv)] Jika diketahui $a+b+c=3$, tunjukkan bahwa $\dfrac{2a}{1+b^2}+\dfrac{2b}{1+c^2}+\dfrac{2c}{1+a^2}\geq 3$. Kapan kesamaan terjadi?
\end{enumerate}

\item Sebanyak $2n$ ($n \geq 2$) koin adil (seimbang dan identik) dilempar. Tunjukkan bahwa peluang munculnya tepat $n$ sisi angka adalah lebih dari $\frac{1}{2^n}$.

\item Diberikan $H$ adalah titik tinggi dari segitiga lancip $ABC$. Misalkan $AD$, $BE$, dan $CF$ adalah garis-garis tinggi segitiga $ABC$, dan $P$ dan $Q$ merupakan perpotongan lingkaran luar segitiga $DEF$ dengan lingkaran luar segitiga $BHC$. Buktikan bahwa $AP = AQ$.

\item Misalkan $a$ dan $b$ adalah bilangan bulat positif dengan $a > b > 1$. Diketahui bahwa \[ a+b\mid ab+1 \text{ dan } a-b\mid ab-1. \] Buktikan bahwa $a < b\sqrt{3}$.

%%% END Bagian B
\end{enumerate}
\end{document}
